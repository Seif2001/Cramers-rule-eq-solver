\documentclass{article}

\usepackage{fancyvrb}
\usepackage[a4paper, total={6in, 8in}]{geometry}
\usepackage[utf8]{inputenc}
\usepackage{amsthm}
\usepackage{amsmath, amssymb}
\usepackage{xcolor}	
\usepackage{tabu}
\usepackage{booktabs}
\usepackage{multirow}
\usepackage{indentfirst}
\usepackage{enumitem}
\usepackage{graphicx}
\theoremstyle{plain}
\newtheorem{theorem}{Theorem}[section]
\newtheorem{corollary}[theorem]{Corollary}
\newtheorem{definition}[theorem]{Definition}
\newtheorem{example}[theorem]{Example}

\parskip=1.9mm
\linespread{1.1}
\setlength\parindent{24pt}


\renewcommand\qedsymbol{$\blacksquare$}
\renewcommand{\restriction}{\mathord{\upharpoonright}}

\title{\textbf{Linear Algebra "Cramer's Rule" Paper\\ ~ \\ MACT2132: Linear Algebra}}
\author{Seifeldin Elshabshiri: \\Allaa ElKhouly:  \\ Salma Ahmed Aly ID: 900203182 \\ ~ \\ The American University in Cairo}
\date{}

\begin{document}
\maketitle
\newpage
\section*{Cramer's Rule}

Cramer's rule is a formula for the solution of a system of linear equations with as many equations as unknowns; it’s only valid when the system has a unique solution. The theorem goes as follow:
Consider the linear system:
\[ a_{11}x_1 + a_{12}x_2 + a_{13}x_3 + ... + a_{1n}x_n = b_1 \]
\[ a_{21}x_1 + a_{22}x_2 + a_{23}x_3 + ... + a_{2n}x_n = b_2 \]
\[ a_{31}x_1 + a_{32}x_2 + a_{33}x_3 + ... + a_{2n}x_n = b_3 \]
\makebox[\textwidth]{\color{white}texttttttttttttttttttttttt\color{black}.\\}
\makebox[\textwidth]{\color{white}texttttttttttttttttttttttt\color{black}.\\}
\makebox[\textwidth]{\color{white}texttttttttttttttttttttttt\color{black}.\\}
\[ a_{n1}x_1 + a_{n2}x_2 + a_{n3}x_3 + ... + a_{nn}x_n = b_n \]
Which in matrix format is:
\[\begin{bmatrix}
a_{11} & a_{12} & a_{13} & ... & a_{1n}\\
a_{21} & a_{22} & a_{23} & ... & a_{2n}\\
a_{31} & a_{32} & a_{33} & ... & a_{3n}\\
.\\
.\\
.\\
a_{n1} & a_{n2} & a_{n3} & ... & a_{nn}
\end{bmatrix}
\begin{bmatrix}
x_1\\
x_2\\
x_3\\
.\\
.\\
.\\
x_n
\end{bmatrix}
=
\begin{bmatrix}
b_1\\
b_2\\
b_3\\
.\\
.\\
.\\
b_n
\end{bmatrix}\]\\
\[x_1 = \frac{D_{x_1}}{D} \color{white}text\color{black} x_2 = \frac{D_{x_2}}{D}\color{white}text\color{black}x_3 = \frac{D_{x_3}}{D}\color{white}text\color{black}...\color{white}text\color{black}x_n = \frac{D_{x_n}}{D}\]
\subsection*{Proof of Cramer's Rule}
We will start by taking an arbitrary 3x3 matrix A and constants b. Where A and b are as follow:
\[A = \begin{bmatrix}
a_{11} & a_{12} & a_{13}\\
a_{21} & a_{22} & a_{23}\\
a_{31} & a_{32} & a_{33}
\end{bmatrix}\color{white}text\color{black}b = \begin{bmatrix}
b_1\\
b_2\\
b_3
\end{bmatrix} \]
The matrix format of a system of linear equations goes as follow:
\[ A x = b\]
\[ x = A^{-1} b\]
\[ x = \frac{1}{det(A)}adj(A) b\]
\[ x = \frac{1}{det(A)} \begin{bmatrix}
C_{11} & C_{21} & C_{31}\\
C_{12} & C_{22} & C_{32}\\
C_{13} & C_{23} & C_{33}
\end{bmatrix} \begin{bmatrix}
b_1\\
b_2\\
b_3
\end{bmatrix}\]
\[=\frac{1}{det(A)}\begin{bmatrix}
C_{11}b_1 + C_{21}b_2 + C_{31}b_3\\
C_{12}b_1 + C_{22}b_2 + C_{32}b_3\\
C_{13}b_1 + C_{23}b_2 + C_{33}b_3
\end{bmatrix} \]
\[x_i = \frac{C_{1i}b_1 + C_{2i}b_2 + C_{3i}b_3}{det(A)}\]
In the aforementioned example, if we expand \[C_{11}b_1 + C_{21}b_2 + C_{31}b_3\] we get:
\[a_{22}a_{33}b_1 - a_{23}a_{32}b_1 + a_{12}a_{33}b_2 - a_{32}a_{13}b_2 + a_{12}a_{23}b_3 - a_{22}a_{13}b_3\]
\[= D_{x_1} = \begin{vmatrix}
b_1 & a_{12} & a_{13}\\
b_1 & a_{22} & a_{23}\\
b_1 & a_{32} & a_{33}
\end{vmatrix}\]
Therefore \[x_i = \frac{D_{x_i}}{D}\]
In this example $x_2$ and $x_3$ are:
\[x_2 = \frac{\begin{vmatrix}
a_{11} & b_1 & a_{13}\\
a_{21} & b_2 & a_{23}\\
a_{31} & b_3 & a_{33}
\end{vmatrix}}{det(A)} = \frac{D_{x_2}}{D} \color{white}texttttttt\color{black}x_3 = \frac{\begin{vmatrix}
a_{11} & a_{12} & b_1\\
a_{21} & a_{22} & b_2\\
a_{31} & a_{32} & b_3
\end{vmatrix}}{det(A)} = \frac{D_{x_3}}{D}\]
\section*{Examples using Cramer's Rule}
\subsection*{Own Example}
Take the following system of linear equations:
\[2x_1 + 3x_2 - 5x_3 = 1\]
\[x_1 + x_2 - x_3 = 2\]
\[2x_2 + x_3 = 8\]
\[D = \begin{vmatrix}
2 & 3 & -5\\
1 & 1 & -1\\
0 & 2 & 1
\end{vmatrix} = -7\]
\[D_{x_1} = \begin{vmatrix}
1 & 3 & -5\\
2 & 1 & -1\\
8 & 2 & 1
\end{vmatrix} = -7\]
\[D_{x_2} = \begin{vmatrix}
2 & 1 & -5\\
1 & 2 & -1\\
0 & 8 & 1
\end{vmatrix} = -21\]
\[D_{x_3} = \begin{vmatrix}
2 & 3 & 1\\
1 & 1 & 2\\
0 & 2 & 8
\end{vmatrix} = -14\]
Therefore the answer will be:
\[x_1 = \frac{-7}{-7} = 1\color{white}texttttttt\color{black}x_2 =\frac{-21}{-7} = 3\color{white}texttttttt\color{black}x_3 =\frac{-14}{-7} = 2\]
\newpage
\section*{References}
\hangindent 

\newpage

\end{document}